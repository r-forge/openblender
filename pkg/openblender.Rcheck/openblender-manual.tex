\nonstopmode{}
\documentclass[a4paper]{book}
\usepackage[times,inconsolata,hyper]{Rd}
\usepackage{makeidx}
\usepackage[utf8]{inputenc} % @SET ENCODING@
% \usepackage{graphicx} % @USE GRAPHICX@
\makeindex{}
\begin{document}
\chapter*{}
\begin{center}
{\textbf{\huge Package `openblender'}}
\par\bigskip{\large \today}
\end{center}
\begin{description}
\raggedright{}
\inputencoding{utf8}
\item[Type]\AsIs{Package}
\item[Title]\AsIs{Request <https://openblender.io> API Services}
\item[Version]\AsIs{0.5.8}
\item[Description]\AsIs{Interface to make HTTP requests to 'OpenBlender' API services. Go to <https://openblender.io> for more information.}
\item[Depends]\AsIs{R (>= 3.3.3), httr (>= 1.4.1), jsonlite (>= 1.5)}
\item[License]\AsIs{MIT + file LICENSE}
\item[Encoding]\AsIs{UTF-8}
\item[LazyData]\AsIs{true}
\item[RoxygenNote]\AsIs{7.0.2}
\item[NeedsCompilation]\AsIs{no}
\item[Author]\AsIs{Open Blender Inc. [cph],
Daniel V. Pinacho [aut, cre]}
\item[Maintainer]\AsIs{Daniel V. Pinacho }\email{danielvpinacho@gmail.com}\AsIs{}
\end{description}
\Rdcontents{\R{} topics documented:}
\inputencoding{utf8}
\HeaderA{call}{Make HTTP request to \R{}href{http://openblender.io}{openblender.io} services}{call}
%
\begin{Description}\relax
Call 'OpenBlender' API services.
\end{Description}
%
\begin{Usage}
\begin{verbatim}
call(action, parameters)
\end{verbatim}
\end{Usage}
%
\begin{Arguments}
\begin{ldescription}
\item[\code{action}] Task you're requesting

\item[\code{parameters}] Request options
\end{ldescription}
\end{Arguments}
%
\begin{Value}
A list that includes the new dataset id in case you create one, success/error message when you insert observations or the list of observations requested.
\end{Value}
%
\begin{SeeAlso}\relax
To see more details go to \Rhref{http://openblender.io}{openblender.io}
\end{SeeAlso}
%
\begin{Examples}
\begin{ExampleCode}
## Not run: 
#CREATE DATASET
df <- read.csv(file = "/path/to/your/data.csv", header = TRUE, sep = ",")
action <- "API_createDataset"
parameters <- list(
token = "YOUR TOKEN",
id_user = "YOUR USER ID",
name = "Assign a name",
descriptipon = "Set a description",
visibility = "public",
tags = list("topic", "tag"),
insert_observations = "off",# set "on" if you want include the observations
dataframe = df
)
call(action, parameters)
#INSERT OBSERVATIONS
df <- read.csv(file = "/path/to/your/data.csv", header = TRUE, sep = ",")
action <- "API_insertObservations"
parameters <- list(
token = "YOUR TOKEN",
id_user = "YOUR USER ID",
id_dataset = "DATASET ID",
notification = "on",
observations = df
)
call(action, parameters)

#GET OBSERVATIONS
action <- "API_getObservationsFromDataset"
parameters <- list(
token = "YOUR TOKEN",
id_user = "YOUR USER ID",
id_dataset = "DATASET ID"
)
call(action, parameters)

## End(Not run)
\end{ExampleCode}
\end{Examples}
\printindex{}
\end{document}
